% !TEX root = main.tex 

\section{Exercises}

\begin{definition}
	Let $\mathcal{A}$ and $\mathcal{B}$ be frames. A family of relations $R^s$ indexed over types $s$ is said to be a logical relation between $\mathcal{A}$ and $\mathcal{B}$ if 
		\begin{itemize}
			\item $R^s \subseteq \mathcal{A} \llbracket s \rrbracket \times \mathcal{B} \llbracket s \rrbracket$
			\item $f R^{s \to t}$ iff $\forall x, y \ x R^s y \implies A^{s,t}(f,x) R^t B^{s,t}(g, y)$
		\end{itemize}
\end{definition}

\begin{ex}{2.16}
	The notion of a partial homomorphism is a special instance of a logical relation. Explain why this is the case and conjecture a version of Lemma 2.22 about partial homomorphisms that would apply to logical relations. Prove your version of the lemma. 
\end{ex} 

\begin{solution}


\begin{definition}
	We say that a binary relation $R \subseteq A \times B$ is \emph{surjective} if 
		\begin{align*}
			\forall b \in B  \ \exists a \in A \ aRb
		\end{align*}
\end{definition}

\begin{definition}
	We say that a binary relation $R \subseteq A \times B $ \emph{function-like} if 
		\begin{align*}
			\forall b \in B \ \forall a_1, a_2 \in A  \ a_1 R b \text{ and } a_1 R B \implies a_1 = a_2 
		\end{align*}
\end{definition}

Observe that every partial homomorphism $\Upphi: \mathcal{A} \to \mathcal{B}$  induces a function-like surjective logical relation $R$ in the following way. For every $\Upphi^s$ define $R^s$ by 
	\begin{align*}
		x R^s y  \Leftrightarrow�\Upphi^s(x) = y 
	\end{align*}
\end{solution}

\begin{ex}{2.15}
	Show that Theorem 2.25 fails when $X$ is not infinite
\end{ex}

\begin{solution}
	For example, if $X$ is a singleton set, then for every $H$-enviroment $\rho$, where $H = x:o, y:o$, we have 
		\begin{align*} 
			\mathcal{F}_X \llbracket H \vartriangleright x:o \rrbracket \rho &= \rho (x) \\
			&= \rho(y) \\
			&= \mathcal{F}_X \llbracket H \vartriangleright y:o \rrbracket \rho 
		\end{align*}
	Thus, 
		\begin{align*}
			\mathcal{F}_X \vDash ( H \vartriangleright x = y : o )
		\end{align*}
	But as we saw in Theorem 2.13, the simply-type $\lambda$-calculus is non-trivial, therefore it is not the case that 
		\begin{align*}
			\vdash (H \vartriangleright x = y :o)
		\end{align*}
	
	For the most general case in which $X$ is finite but not necessarily a singleton set, consider the following Church encoding of the natural numbers 
	\begin{align*}
		c_0 &= \lambda f : o \to o. \lambda x : o.  \ x \\
		c_n &= \lambda f:o \to o. \lambda x:o. \  f(c_{n -1} f x), \ n \geq 1 
	\end{align*}

	Observe that if $X_1$ and $X_2$ are finite, then so it is $X_2^{X_1}$, the set of functions from $X_1$ to $X_2$. Thus, for every type $t$, $\mathcal{F}_X \llbracket t \rrbracket$ must be finite. 
	
	Observe also that since each $c_n$ is a closed term, the denotation on $\mathcal{F}_X$ will be independent of the environment, that is, for any two environments $\rho_1$ and $\rho_2$ 
		\begin{align*}
			\mathcal{F}_X \llbracket \vartriangleright c_n: t \rrbracket \rho_1 = \mathcal{F}_X \llbracket \vartriangleright c_n : t \rrbracket \rho_2 
		\end{align*}
		for each $n \in \mathbb{N}$, where $t = (( o \to o) \to o) \to o$. 
	
	Since we can choose finitely many elements out of $\mathcal{F}_X \llbracket t \rrbracket$, then it must be the case that for some $n, m \in \mathbb{N}$ with $n < m$ we have 
		\begin{align*}
			\mathcal{F}_X \vDash ( \vartriangleright c_n = c_n : t )
		\end{align*}
	
	But clearly that fails to hold in every frame. For example, in $\mathcal{F}_\mathbb{N}$, if we let $s: \mathbb{N} \to \mathbb{N}$ be the successor function, then 
		\begin{align*}
			(\mathcal{F}_\mathbb{N}�\llbracket \vartriangleright c_n : t \rrbracket \rho s) 0 &= n \\
				& \neq m \\
				& = (\mathcal{F}_\mathbb{N}�\llbracket \vartriangleright c_m : t \rrbracket \rho s) 0
		\end{align*}
	Thus, 
		\begin{align*}
			\mathcal{F}_\mathbb{N} \nvDash ( \vartriangleright c_n = c_m : t )
		\end{align*} 
	and, consequently, the soundness result implies that 
		\begin{align*}
			\nvdash ( \vartriangleright c_n = c_m : t )
		\end{align*}
	Hence, Theorem 2.25 fails. 
\end{solution}




















